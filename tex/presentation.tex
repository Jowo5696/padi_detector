\documentclass[t,9pt]{beamer}

\usetheme{CambridgeUS}
\usecolortheme{seahorse}

\setbeamertemplate{navigation symbols}{}
\setbeamertemplate{section in toc}[ball unnumbered]
\setbeamertemplate{headline}{}
 
\usefonttheme{serif}
\usepackage{newtxmath}
\usepackage[onehalfspacing]{setspace}

% \usepackage[ngerman]{babel}
\usepackage{amsmath,amssymb,amsfonts}
\usepackage{siunitx}
\usepackage[absolute,overlay]{textpos}
\usepackage{bookmark}
\usepackage{csquotes}
\usepackage{subcaption}

\usepackage{graphicx}%
\usepackage{multirow}%
\usepackage{amsthm}%
\usepackage{physics}
\usepackage{mathrsfs}%
\usepackage[title]{appendix}%
\usepackage{xcolor}%
\usepackage{textcomp}%
\usepackage{manyfoot}%
\usepackage{booktabs}%
\usepackage{algorithm}%
\usepackage{algorithmicx}%
\usepackage{algpseudocode}%
\usepackage{listings}%
\usepackage{enumitem}
\usepackage{appendix}
\usepackage{hyperref} 

\usepackage{caption}
\usepackage{svg}

\usepackage[framemethod=TikZ]{mdframed}
\usepackage{tcolorbox}
\tcbuselibrary{theorems}
\tcbuselibrary{skins}

\newcommand{\td}{\text{d}}

\newcommand{\highlight}[3]{ \begin{textblock*}{#1}(#2,#3) \begin{tcolorbox} [enhanced,opacityfill=.1,colback=blue] \end{tcolorbox} \end{textblock*} } % \highlight{100pt}{10pt}{25pt}

\title[\thesection]{Particle Detectors and Instrumentations}
\subtitle{Observation of Multiple Scattering}
\author{Group 1}
\institute{Universität Bonn}
\date{\today}
%\logo{\LaTeX{}}

\begin{document}

        \iffalse\AddToHook{shipout/foreground}{
          \begin{tikzpicture}[remember picture,overlay]
            \node[red,rotate=30,scale=5,opacity=0.1] at (current page.center) {Draft};
          \end{tikzpicture}
        }\fi

        \begin{frame}
                \titlepage
        \end{frame}

        % \begin{frame}{Inhaltsverzeichnis}
        %         \tableofcontents
        % \end{frame}

        \begin{frame}
                \vfill
                \begin{figure}
                        \centering
                        \includegraphics[trim={5cm 0 5cm 0},clip, width=.9\textwidth]{./figures/detector_setup_elsa.jpg}
                        \caption{Setup of experimental site at ELSA.}
                \end{figure}
        \end{frame}

        \begin{frame}
                \vfill
                \begin{figure}
                        \centering
                        \includegraphics[width=.8\textwidth]{../src/elsa/finished_plots/xy_hitmap_0.png}
                        \caption{Hitmap of the beam without target.}
                \end{figure}
        \end{frame}

        \begin{frame}
                \vfill
                \begin{figure}
                        \centering
                        \includegraphics[width=.8\textwidth]{../src/elsa/finished_plots/UnfilteredNoMaterialX.png}
                        \caption{Raw data of the beam along the x-axis without target.}
                \end{figure}
        \end{frame}

        \begin{frame}
                \vfill
                \begin{figure}
                        \centering
                        \includegraphics[width=.8\textwidth]{../src/elsa/finished_plots/unfiltered_noMaterial.png}
                        \caption{Raw data of the beam along the y-axis without target.}
                \end{figure}
        \end{frame}

        \begin{frame}
                \vfill
                \begin{figure}
                        \centering
                        \includegraphics[width=.8\textwidth]{../src/elsa/finished_plots/no target.png}
                        \caption{Filtered beam profile along y-axis without target.}
                \end{figure}
        \end{frame}

        \begin{frame}
                \begin{figure}
                        \centering
                        \begin{tikzpicture}
                                \node (img1) {\includegraphics[width=.4\textwidth]{../src/elsa/finished_plots/Aluminium, Half Radiation Length, 40cm Distance.png}};
                                \node (img2) at (img1.east) [xshift = 3cm] {\includegraphics[width=.4\textwidth]{../src/elsa/finished_plots/Aluminium, One Radiation Length, 40cm Distance.png}};
                                \node (img3) at (img1.south) [yshift = -2cm] {\includegraphics[width=.4\textwidth]{../src/elsa/finished_plots/Aluminium, Two Radiation Lengths, 40cm Distance.png}};
                                \node (img4) at (img3.east) [xshift = 3cm] {\includegraphics[width=.4\textwidth]{../src/elsa/finished_plots/Aluminium, Three Radiation Lengths, 40cm Distance.png}};
                        \end{tikzpicture}
                        \caption{Counted hits for aluminium of different thicknesses $x$.}
                \end{figure}
        \end{frame}

        \begin{frame}
                \vfill
                \begin{figure}
                        \centering
                        \includegraphics[width=.8\textwidth]{../src/elsa/finished_plots/Aluminium.png}
                        \caption{Fit of experimental $\theta_{\text{exp}}$ values against theoretical values of $\theta_{0}$ for aluminium}
                \end{figure}
        \end{frame}

        \begin{frame}
                \begin{figure}
                        \centering
                        \begin{tikzpicture}
                                \node (img1) {\includegraphics[width=.4\textwidth]{../src/elsa/finished_plots/Copper, Half Radiation Length, 40cm Distance.png}};
                                \node (img2) at (img1.east) [xshift = 3cm] {\includegraphics[width=.4\textwidth]{../src/elsa/finished_plots/Copper, One Radiation Length, 40cm Distance.png}};
                                \node (img3) at (img1.south) [yshift = -2cm] {\includegraphics[width=.4\textwidth]{../src/elsa/finished_plots/Copper, Two Radiation Lengths, 40cm Distance.png}};
                                \node (img4) at (img3.east) [xshift = 3cm] {\includegraphics[width=.4\textwidth]{../src/elsa/finished_plots/Copper, Three Radiation Lengths, 40cm Distance.png}};
                                \node (capt1) at (img1.south) {$x\approx\frac{X_0}{2}$};
                        \end{tikzpicture}
                        \caption{Counted hits for copper of different thicknesses $x$.}
                \end{figure}
        \end{frame}

        \begin{frame}
                \vfill
                \begin{figure}
                        \centering
                        \includegraphics[width=.8\textwidth]{../src/elsa/finished_plots/Copper.png}
                        \caption{Fit of experimental $\theta_{\text{exp}}$ values against theoretical values of $\theta_{0}$ for copper}
                \end{figure}
        \end{frame}

%        \begin{frame}
%                \vfill
%                \begin{table}
%                        \begin{tabular}{ccc}
%                                \toprule
%                                $\tfrac{x}{X_0}$ & $\theta _\text{exp}$/rad & $\theta _\text{theo}$/rad \\
%                                \midrule
%                                1/2 & \num{.00812+-.00343}[42\%,1.3$\sigma$] & 0.0343\\
%                                1 & \num{.01274+-.00198}[16\%,3.9$\sigma$] & 0.00497\\
%                                2 & \num{.01961+-.00593}[30\%,2$\sigma$] & 0.00744\\
%                                3 & \num{.02329+-.00632}[27\%,2.3$\sigma$] & 0.0858\\
%                                \midrule
%                                1/2 & \num{.00778+-.00678}[87\%,0.7$\sigma$] & 0.00318\\
%                                1 & \num{.01187+-.00171}[14\%,4.2$\sigma$] & 0.00476\\
%                                2 & \num{.01995+-.00510}[26\%,2.6$\sigma$] & 0.00692\\
%                                3 & \num{.02129+-.00592}[28\%,2.2$\sigma$] & 0.00851\\
%                                \bottomrule
%                        \end{tabular}
%                        \caption{An overview of the experimentally and theoretically calculated angles; aluminium top and copper bottom. 
%                        The relative error and standard deviation from the theoretical values is given in the square brackets.}
%                \end{table}
%        \end{frame}

\end{document}